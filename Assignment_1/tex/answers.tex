%!TEX root=assignment_submission.tex
\section*{Question 1:}
\subsection*{Question 1.1}
$Average\;Precision = \frac{1}{|R|} \Sigma_{i=1}^n prec(i)rel(i)$
\subsubsection*{List 1: \textbf{\textit{A, C, E, D}}}
$ap = \frac{1}{3} \big(1+ 0 + \frac{2}{3} + 0\big) = \frac{1}{3}\big(\frac{5}{3}\big) = \frac{5}{9}$

\subsubsection*{List 2: \textbf{\textit{C, B, F, D}}}
$ap = \frac{1}{3} \big(0 + \frac{1}{2} + 0 + 0\big) = \frac{1}{6}$

\subsubsection*{List 3: \textbf{\textit{E, D, C, F}}}
$ap = \frac{1}{3} \big(1 + 0 + 0 + 0\big) = \frac{1}{3}$

\subsubsection*{List 4: \textbf{\textit{A, D, C, E}}}
$ap = \frac{1}{3} \big(1+ 0 + 0 + \frac{2}{4}\big) = \frac{1}{3}\big(\frac{6}{4}\big) = \frac{6}{12} = \frac{1}{2}$

\subsubsection*{List 4: \textbf{\textit{F, A, C, B}}}
$ap = \frac{1}{3} \big(0 + \frac{1}{2} + 0 + \frac{2}{4}\big) = \frac{1}{3}\big(1\big)= \frac{1}{3}$

\subsection*{Question 1.2}
Normalized Discounted Cumulative Gain
$$
nDCG = \frac{DCG}{IDCG}
$$
where DCG is the Discounted Cumulative Gain
$$
DCG = \Sigma_{i=1}^n \frac{rel(i)}{\log_2(i+1)}
$$
and IDCG is the Ideal Discounted Cumulative Gain.\\\\
The ideal list for this query would be \textbf{\textit{A, B, E, C}}
$$
IDCG = 4 + \frac{2}{\log_2(3)} + \frac{1}{\log_2(4)} = 5.7618
$$

\subsubsection*{List 1: \textbf{\textit{A, C, E, D}}}
$nDCG = \frac{4 + 0 + \frac{1}{\log_2(4)} + 0}{5.7618} = \frac{4.5}{5.7618} = 0.781$

\subsubsection*{List 2: \textbf{\textit{C, B, F, D}}}
$nDCG = \frac{0 + \frac{2}{log_2(3)} + 0 + 0}{5.7618} = \frac{1.262}{5.7618} = 0.219$

\subsubsection*{List 3: \textbf{\textit{E, D, C, F}}}
$nDCG = \frac{1 + 0 + 0 + 0}{5.7618} = \frac{1}{5.7618} = 0.174$

\subsubsection*{List 4: \textbf{\textit{A, D, C, E}}}
$nDCG = \frac{4 + 0 + 0 + \frac{1}{\log_2(5)}}{5.7618} = \frac{4.4307}{5.7618} = 0.769$

\subsubsection*{List 4: \textbf{\textit{F, A, C, B}}}
$nDCG = \frac{0 + \frac{4}{\log_2(3)} + 0 + \frac{2}{\log_2(5)}}{5.7618} = \frac{3.3851}{5.7618} = 0.588$ 

\section*{Question 2: }

\subsection*{Question 2.1 }
\subsubsection*{Tokenization}
Tokenization splits the sentence into individual tokens. Tokens can be words, name entities (i.e. people's names, city names, etc), or Email Addresses, URLs, etc. Punctuation's are also removed. Below each token is in its own cell.
\begin{center}
\begin{tabular}{| c | c | c | c | c | c | c |} \hline
According & to & Wikipedia & Information & Retrieval & is & the \\ \hline
activity & of & obtaining & information & resources & relevant & to \\ \hline
an & information & need & from & a & collection & of \\ \hline
information & resource & & & & & \\ \hline
\end{tabular}
\end{center}

\subsubsection*{Normalization}
Normalization transforms text into a single canonical form such a lower-casing or removing whitespace.
\begin{center}
\begin{tabular}{| c | c | c | c | c | c | c |} \hline
according & to & wikipedia & information & retrieval & is & the \\ \hline
activity & of & obtaining & information & resources & relevant & to \\ \hline
an & information & need & from & a & collection & of \\ \hline
information & resource & & & & & \\ \hline
\end{tabular}
\end{center}

\subsubsection*{Stopping}
Stopping removes stopwords. Stopwords appear frequently in text, but are usually uninformative.
The words `to', `is', `the', `of', `an', `from', and `a' were chosen as stopwords because they appear frequently in text and are non-informative.

\begin{center} 
\begin{tabular}{| c | c | c | c | c | c | c |} \hline
according &  & wikipedia & information & retrieval &  &  \\ \hline
activity &  & obtaining & information & resources & relevant &  \\ \hline
 & information & need & &  & collection &  \\ \hline
information & resource & & & & & \\ \hline
\end{tabular}
\end{center}

\subsubsection*{Krovetz Stemming}
Stemming reduces inflected or derived words to their word stem, base or, root, i.e. plural to singular and normalizes verb tense.

\begin{center} 
\begin{tabular}{| c | c | c | c | c | c | c |} \hline
accord &  & wikipedia & information & retrieveal &  &    \\ \hline
activity &  & obtain & information & resource & relevant &    \\ \hline
 & information & need & &  & collection &    \\ \hline
information & resource & & & & &   \\ \hline
\end{tabular}
\end{center}

\subsection*{Question 2.2}
\subsubsection*{What is advantage of searching with inverted index compared to searching by sequentially reading each document?}
The time, memory, and processing is much lower; and in the case of the web or large databases of documents the inverted index makes the search feasible.

\subsubsection*{Does an inverted index improve the efficiency of a search system in all cases? If so, explain why; if not, give an example.}
No, the updates time (adding a document) for an inverted index is much slower.

\subsection*{Question 2.3}
\subsubsection*{Encode the number 646 with both $\gamma$-code and $\delta$-code}
\subsubsection*{$\gamma$-code}
$$
x_d = \lfloor\log_2(646)\rfloor = 9
$$
$$
x_r = 646 - 2^{\lfloor\log_2(646)\rfloor} = 134
$$
Representing $x_d$ as unaray and $x_r$ as binary we get: $0000000001\;010000110$.
\subsubsection*{$\delta$-code}
$$
x_d = \lfloor\log_2(646)\rfloor = 9
$$
$$
x_{dd} = \lfloor\log_2(9 + 1)\rfloor = 3
$$
$$
x_{dr} = (9 + 1) - 2^{\lfloor\log_2(9 + 1)\rfloor} = 10 - 2^3 = 2
$$
$$
x_{r} = 646 - 2^{\lfloor\log_2(646)\rfloor} = 646 - 2^9 = 134
$$
Representing $x_{dd}$ as unaray and $x_{dr}$ and $x_r$ as binary we get: $0001 \; 010 \; 10000110$.

\subsubsection*{Determine encoding and decode $\textbf{0001010}$ and $\textbf{001010101}$}
\subsubsection*{$\textbf{0001010}$}
This is $\gamma$-code.
$$
x_d = 0001 = 3
$$
$$
x_r = 010 = 2
$$
$$
2 = x - 2^3
$$
$$
x = 10
$$

\subsubsection*{$\textbf{001010101}$}
This is $\delta$-code
$$
x_{dd} = 001 = 2
$$
$$
x_{dr} = 01 = 1
$$
$$
x_r = 0101 = 5
$$
Solving for $x_d$
$$
2 = \log_2(x_d + 1)
$$
$$
2^2 = x_d + 1
$$
$$
4 - 1 = x_d
$$
$$
x_d = 3
$$
Solving for $x$
$$
5 = x - 2^3
$$
$$
5 = x - 8
$$
$$
5 + 8 = x
$$
$$
x = 13
$$

\subsection*{Experimental Question}
\subsubsection*{Task 1: PageRank}
\begin{center} 
\begin{tabular}{| c | c |} \hline
\textbf{Document ID} & \textbf{PageRank Score}  \\ \hline
1 & 1.03773405 \\ \hline
2 & 0.62264043 \\ \hline
3 & 0.0 \\ \hline
4 & 0.41509362 \\ \hline
5 & 0.83018724 \\ \hline
6 & 0.0 \\ \hline
7 & 0.20754681 \\ \hline        
\end{tabular}
\end{center}

\subsubsection*{Task 2: Indexing}